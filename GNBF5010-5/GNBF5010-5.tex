\documentclass[UTF8]{beamer}
\usepackage{graphicx, color}
\usepackage{algorithm2e}
\usepackage{zhspacing}
\usepackage{amsmath}

\usepackage{underscore}
\usetheme{JuanLesPins}
\usepackage{fontspec}
\setsansfont{Microsoft YaHei}

\usepackage{enumerate}

\AtBeginSection[]{
  \frame{
    \frametitle{Next}
    \tableofcontents[currentsection, subsectionstyle=show/shaded/hide]
  }
}

\AtBeginSubsection[]{
  \frame{
    \frametitle{Next}
    \tableofcontents[currentsubsection]
  }
}

\title{Perl in Bioinformatics}

\author{Gang Chen\\ chengang@bgitechsolutions.com}

\logo{\includegraphics[width=1.3cm]{bgi-logo.png}\includegraphics[width=2.5cm]{cuhklogo.png}}
\date{\today}




\begin{document}


\begin{frame}
\titlepage
\end{frame}
\begin{frame}[t]\frametitle{Outline}
\tableofcontents[hideallsubsections]
\end{frame}

\section{Perl Modules}

\subsection{Usage of Perl Modules}

\begin{frame}
  \frametitle{Web Scraping in Perl}
  \begin{block}{LWP}
    The libwww-perl collection is a set of Perl modules which provides a simple
    and consistent application programming interface (API) to the World-Wide Web.

    https://metacpan.org/pod/LWP
  \end{block}
\end{frame}

\begin{frame}
  \frametitle{Example: lwp.pl}
\begin{block}{Files}
  \begin{itemize}
    \item pubmedids.txt: A list of pubmed ids
    \item lwp.pl: Get publication titles of these ids from PubMed.
  \end{itemize}
\end{block}
\end{frame}

\subsection{Management of Perl Modules}

\begin{frame}
  \frametitle{Installation}
\end{frame}



\subsection{Development of Perl Modules}

\begin{frame}
  \frametitle{Development}
  see PUBMED.pm and run.pl
\end{frame}

\subsection{Submit your modules to CPAN}

\begin{frame}
  \frametitle{Why?}
  \begin{itemize}
    \item Easy the installation of your module
    \item Share your work with the community
    \item Increase your importance in the community
    \item Get response form the community to improve your module
  \end{itemize}
\end{frame}

\begin{frame}
  \frametitle{How?}
  http://www.cpan.org/modules/04pause.html
\end{frame}

\section{SNP Annotation}

\subsection{A Perl Script for SNP Annotation}

\subsection{SNPAnno module}

\section{Circos}

\section{BioPerl}

\end{document}
