\documentclass[UTF8]{beamer}
\usepackage{graphicx, color}
\usepackage{algorithm2e}
\usepackage{zhspacing}
\usepackage{amsmath}
\usepackage{tikz}
\usepackage{url}
\usetikzlibrary{shapes,arrows}

% Define block styles
\tikzstyle{decision} = [diamond, draw, fill=blue!20,
    text width=4.5em, text badly centered, node distance=3cm, inner sep=0pt]
\tikzstyle{block} = [rectangle, draw, fill=blue!20,
    text width=5em, text centered, rounded corners, minimum height=3em]
\tikzstyle{line} = [draw, -latex']
\tikzstyle{cloud} = [draw, ellipse,fill=red!20, node distance=3cm,
    minimum height=2em]

\usepackage{underscore}
\usetheme{JuanLesPins}
\usepackage{fontspec}
\setsansfont{Microsoft YaHei}

\usepackage{enumerate}

\AtBeginSection[]{
  \frame{
    \frametitle{Next}
    \tableofcontents[currentsection, subsectionstyle=show/shaded/hide]
  }
}

\AtBeginSubsection[]{
  \frame{
    \frametitle{Next}
    \tableofcontents[currentsubsection]
  }
}

\title{Introduction to C}

\subtitle{Development Environment and Quick Get Started}

\author{Gang Chen\\ chengang@genomics.cn}

\logo{\includegraphics[width=1.3cm]{bgi-logo.png}\includegraphics[width=2.5cm]{cuhklogo.png}}
\date{\today}




\begin{document}

\begin{frame}
\titlepage
\end{frame}

\begin{frame}[t]{Assignment of C}
\begin{enumerate}
    \item Download and install a C compiler on your computer;
    \item Write a program to print Fibonacci sequence.The length of output sequence is specified by the first command line parameter. (fibonacci.c)
    \item Implement Smith-Waterman algorithm in C. Given that the cost of GAP, MATCH and MISMATCH is -1, 2 and 0.5, separately. Calculate the alignment of “ACGTGGCCTTGTGA” and “GGTGGGTCTTGTCG”.
\end{enumerate}


\end{frame}
%--- Next Frame ---%

\begin{frame}[t]\frametitle{Outline}
\tableofcontents[hideallsubsections]
\end{frame}

\section{Overview}
\subsection{What is C?}
\begin{frame}[t]{C}
C is widely used in various environments, including network programming, operating systems, implementing programming language, embedded devices, high performance numerical computing and so on.

In this course, all other programming languages, Java, Python, Perl and R, are based on the C programming language.

C is fundamental to modern computer software, including bioinformatics software.

\end{frame}
%--- Next Frame ---%


\section{Get Started}
\subsection{C Compilers}
\begin{frame}[t]{C Compilers}
    \begin{itemize}
        \item gcc in GCC: GNU Compiler Collection
        \item Clang in LLVM
        \item Microsoft Visual C++
        \item Intel C++ Compiler
        \item Turbo C from Borland
        \item List: \url{https://en.wikipedia.org/wiki/List_of_compilers\#C_compilers}
    \end{itemize}
\end{frame}
%--- Next Frame ---%

\begin{frame}[t]{gcc}
    \begin{itemize}
        \item http://gcc.gnu.org/
        \item GCC 5.2, GCC 4.9.3
        \item The GNU Compiler Collection includes front ends for C, C++, Objective-C, Fortran, Java, Ada, and Go, as well as libraries for these languages.
    \end{itemize}

\end{frame}
%--- Next Frame ---%

\begin{frame}[t]{Clang}
    \begin{itemize}
        \item http://clang.llvm.org/
        \item Supported by Apple
        \item The goal of the Clang project is to create a new C, C++, Objective C and Objective C++ front-end for the LLVM compiler.
    \end{itemize}
\end{frame}
%--- Next Frame ---%

\subsection{Download and Install}
\begin{frame}[t]{Linux}
    \begin{itemize}
        \item GCC: Most linux distributions are shipped with GCC
        \item Clang: Pre-Built for Fedora, OpenSuSE, AArch and Ubuntu
        \item GCC is recommeded for Linux
    \end{itemize}
\end{frame}
%--- Next Frame ---%

\begin{frame}[t]{Mac OS}
    \begin{itemize}
        \item LLVM:
        \begin{itemize}
            \item included in Command Line Tools OS X from Apple
            \item download from \url{http://llvm.org/releases/download.html}
        \end{itemize}
    \end{itemize}
\end{frame}
%--- Next Frame ---%

\begin{frame}[t]{Windows}
    \begin{itemize}
        \item Clang provides pre-built version for Windows
        \item GCC for windows is included in MinGW (\url{http://mingw.org/})
    \end{itemize}
\end{frame}
%--- Next Frame ---%

\subsection{Hello World!}

\begin{frame}[t,fragile]{Hello World!}
    \begin{verbatim}
#include <stdio.h>

int main(){

	printf(“Hello!\n”);

	int a = 1, b = 2;
	printf(“The sum of a and b is %i\n”, a+b);

	return 0; // return 0 to system

}
    \end{verbatim}
    hello.c
\end{frame}
%--- Next Frame ---%

\begin{frame}[t, fragile]{Compile and Execute}
    \begin{verbatim}
gcc hello.c -o hello
./hello
Hello
    \end{verbatim}
\end{frame}
%--- Next Frame ---%

\begin{frame}[t,fragile]{Hello World!}
    \begin{verbatim}
#include <stdio.h> // import library for I/O

int main(){ // define main function

	printf(“Hello!\n”);   // print something to the screen

	 int a = 1, b = 2;    // define two variables and assign values
	printf(“The sum of a and b is %i\n”, a+b);
	// print the sum of the two variables to the screen

	return 0; // return 0 to system

}
    \end{verbatim}
\end{frame}
%--- Next Frame ---%

\section{Syntax}
\begin{frame}[t]{Variable and Data Type}
    \begin{itemize}
        \item char
        \item int
        \item float
        \item double
        \item array
        \item pointer
    \end{itemize}
    variables.c
\end{frame}
%--- Next Frame ---%

\begin{frame}[t]{Operations}
    \begin{itemize}
        \item +, \-, *, \/, \%
        \item >, >=, <, <=
        \item ==, !=
        \item ++, \-\-
        \item !
        \item \&\&, ||
    \end{itemize}
    operations.c
\end{frame}
%--- Next Frame ---%

\begin{frame}[t]{Puzzle}
    \begin{itemize}
        \item 1.3 - 0.7 = 0.600000
        \item 1.3 - 0.7 == 0.6 is false
        \item 1.3 - 0.7 != 0.6 is true
    \end{itemize}
    operations.c
\end{frame}
%--- Next Frame ---%

\begin{frame}[t]{Conditional Statements}
    \begin{itemize}
        \item if-else
        \item switch
    \end{itemize}
    flow.c
\end{frame}
%--- Next Frame ---%

\begin{frame}[t]{loop statement}
    \begin{itemize}
        \item \textit{while}
        \item for
        \item break and continue
    \end{itemize}
    flow.c
\end{frame}
%--- Next Frame ---%

\begin{frame}[t,fragile]{Function}
\begin{verbatim}
int add(int a, int b){
    return a+b;
}
\end{verbatim}
    function.c
\end{frame}
%--- Next Frame ---%

\begin{frame}[t,fragile]{Struct}
\begin{verbatim}
struct point{
    int x;
    int y;
}
\end{verbatim}
struct.c
\end{frame}
%--- Next Frame ---%


\section{Libraries}
\begin{frame}[t]{Input/Output}
    stdio.h
    \begin{itemize}
        \item printf
        \item scanf
        \item fopen
        \item fprintf and fscanf
    \end{itemize}
\end{frame}
%--- Next Frame ---%

\begin{frame}[t]{Math}
    math.h
    \begin{itemize}
        \item sin, cos, tan, asin \ldots
        \item exp, log, log10 \ldots
        \item pow, sqrt \ldots
        \item floor, ceil, fabs \ldots
    \end{itemize}
    math.c
\end{frame}
%--- Next Frame ---%

\begin{frame}[t]{String}
    string.h
    \begin{itemize}
        \item strcpy
        \item strcat
        \item strcmp
        \item strlen
    \end{itemize}
\end{frame}
%--- Next Frame ---%

\section*{Summary}
\begin{frame}[t]{Summary}
    Summary
\begin{enumerate}
    \item C is fundamental to modern computer software.
    \item Almost everything is based on C, including most programming languages.
    \item C is a simple, cross-platform and efficient programming language.
    \item C is widely used the development of various software, from small tools and big data systems.
\end{enumerate}
\end{frame}
%--- Next Frame ---%
\begin{frame}
  \centerline{\Huge{Thanks!}}
\end{frame}
%--- Next Frame ---%

\end{document}
