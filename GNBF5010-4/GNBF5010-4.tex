\documentclass[UTF8]{beamer}
\usepackage{graphicx, color}
\usepackage{algorithm2e}
\usepackage{zhspacing}
\usepackage{amsmath}

\usepackage{underscore}
\usetheme{JuanLesPins}
\usepackage{fontspec}
\setsansfont{Microsoft YaHei}

\usepackage{enumerate}

\AtBeginSection[]{
  \frame{
    \frametitle{Next}
    \tableofcontents[currentsection, subsectionstyle=show/shaded/hide]
  }
}

\AtBeginSubsection[]{
  \frame{
    \frametitle{Next}
    \tableofcontents[currentsubsection]
  }
}

\title{Introduction to Perl}

\author{Gang Chen\\ chengang@bgitechsolutions.com}

\logo{\includegraphics[width=1.3cm]{bgi-logo.png}\includegraphics[width=2.5cm]{cuhklogo.png}}
\date{\today}




\begin{document}


\begin{frame}
\titlepage
\end{frame}
\begin{frame}[t]\frametitle{Outline}
\tableofcontents[hideallsubsections]
\end{frame}

\section{Overview}

\begin{frame}[t]{What is Perl?}
\begin{block}{Perl}
  \begin{itemize}
    \item Practical Extraction and Report Language
    \item Pathologically Eclectic Rubbish Lister
  \end{itemize}
\end{block}
\end{frame}
%--- Next Frame ---%

\begin{frame}[t]{Perl: History 1}
\begin{columns}
  \begin{column}{.7\textwidth}
\begin{itemize}
  \item 1.0: December 18, 1987, Larry Page
  \item 2.0: 1988, a better regular expression
  \item 3.0: 1989, support binary data streams
  \item 4.0, 1991
  \item Programming Perl, Camel Book, for Perl 4.0
\end{itemize}
\end{column}
\begin{column}{.3\textwidth}
\includegraphics[width=\textwidth]{programmingperl.jpg}
\end{column}
\end{columns}
\end{frame}
%--- Next Frame ---%

\begin{frame}[t]{Perl: History 2}
\begin{block}{Perl 5}
  \begin{itemize}
    \item 5.000: October 17, 1994, rewrite of the interpreter\\
    Objects, lexical variables, modules and references are added.
    \item 5.002: new prototypes feature.
    \item Comprehensive Perl Archive Network(CPAN), 1995.
    \item 5.004: May 15, 1997, UNIVERSAL package and CGI.pm module.
    \item 5.8: July 18, 2002, unicode, a new I/O, thread
    \item 5.10: December 18, 2007
    \item 5.20: May 27, 2014, subroutin signature, slice.
  \end{itemize}
\end{block}
\end{frame}
%--- Next Frame ---%

\begin{frame}[t]{Perl: History 3}
\begin{block}{Perl 6}
  \begin{itemize}
    \item Perl 6 design process was first announced on July 19, 2000
    \item As of 2014, none of Perl 6 implementation are considered ``complete''.
    \begin{itemize}
      \item Rakudo Perl: Perl 6 for virtual machines.
      \item Pugs: Perl 6 written in Haskell.
      \item v6.pm: a pure Perl 5 implementation of Perl 6.
      \item Yapsi: a Perl 6 compiler and runtime written in Perl 6 itself.
    \end{itemize}
  \end{itemize}
\end{block}
\end{frame}
%--- Next Frame ---%

\begin{frame}[t]{Applications}
\begin{block}{Applications}
  \begin{itemize}
    \item text processing
    \item CGI programming: Craigslist, IMDb, Slashdot and so on;
    \item graphics programming: Perl/Tk, WxPerl
    \item system administration
    \item network programming
    \item bioinformatics
  \end{itemize}
\end{block}
\end{frame}
%--- Next Frame ---%

\begin{frame}[t]{References}
\begin{block}{Books}
  \begin{itemize}
    \item Learning Perl sixth Edition;
    \item Mastering Perl;
    \item Advanced Perl;
    \item Programming Perl;
  \end{itemize}
\end{block}
\begin{block}{Official Website}
  http://www.perl.org/
\end{block}
\end{frame}
%--- Next Frame ---%

\section{Quick Get Started}

\begin{frame}[t]{Install}
\begin{block}{Download and Install}
  Download: http://www.perl.org/get.html
  \begin{itemize}
    \item Unix/Linux: preinstalled
    \item Mac OS: preinstalled
    \item Windows:
    \begin{itemize}
      \item ActiveState Perl: A binary distribution for Win
      \item StrawBerry Perl: Open source
      \item DWIM Perl: based on StrawBerry and include many useful CPAN modules
    \end{itemize}
  \end{itemize}
\end{block}
\end{frame}
%--- Next Frame ---%

\begin{frame}[t]{Install: editor}
\begin{block}{Editors}
  \begin{itemize}
    \item Vim: editor god
    \item Emacs: god's editor
    \item Notepad++: fast and easy to use
    \item Atom, sublime text, textmate \ldots
  \end{itemize}
\end{block}
\end{frame}
%--- Next Frame ---%


\begin{frame}[t]{Hello Perl!}
  \centerline{\includegraphics[height=.6\textheight]{hello.png}}
  see hello.pl
\end{frame}
%--- Next Frame ---%

\begin{frame}[t]{Input and Run}
  \begin{enumerate}
    \item Input the source codes by using a editor
    \item Save the source codes to a file named hello.pl
    \item Execute the file:
    \begin{itemize}
      \item Add execution permission to the file and execute directly
      \item Execute the file by using perl interpreter
    \end{itemize}
  \end{enumerate}
\end{frame}
%--- Next Frame ---%

\section{Syntax}
\subsection{Basic Syntax}
\subsection{Regular Expression}
\subsection{Input and Output}
\subsection{Object Oriented Programming}

\section{Examples}

\subsection{System Administration}

\subsection{String Processing}

\subsection{CGI Programming}

\begin{frame}
  \centerline{\Huge{Thanks!}}
\end{frame}
%--- Next Frame ---%

\end{document}
