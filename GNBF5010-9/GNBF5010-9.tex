\documentclass[UTF8]{beamer}
\usepackage{graphicx, color}
\usepackage{algorithm2e}
\usepackage{zhspacing}
\usepackage{amsmath}

\usepackage{underscore}
\usetheme{JuanLesPins}
\usepackage{fontspec}
\setsansfont{Microsoft YaHei}

\usepackage{enumerate}

\AtBeginSection[]{
  \frame{
    \frametitle{Next}
    \tableofcontents[currentsection, subsectionstyle=show/shaded/hide]
  }
}

\AtBeginSubsection[]{
  \frame{
    \frametitle{Next}
    \tableofcontents[currentsubsection]
  }
}

\title{Python in Bioinformatics}

\author{Gang Chen\\ chengang@bgitechsolutions.com}

\logo{\includegraphics[width=1.3cm]{bgi-logo.png}\includegraphics[width=2.5cm]{cuhklogo.png}}
\date{\today}




\begin{document}

\begin{frame}
\titlepage
\end{frame}

\begin{frame}[t]\frametitle{Outline}
\tableofcontents[hideallsubsections]
\end{frame}

\section{Python Package}
\begin{frame}
  \frametitle{Python Package}
\end{frame}

\begin{frame}
  \frametitle{Python Package Development}
\end{frame}

\begin{frame}
  \frametitle{Example}
  see hello directory
\end{frame}

\section{Python based Bioinformatics Projects}
\subsection{Scipy}
\begin{frame}
  \frametitle{scipy project}
\end{frame}

\begin{frame}
  \frametitle{Installation and Example}
\end{frame}
\subsection{BioPython}
\begin{frame}
  \frametitle{BioPython}
\end{frame}

\begin{frame}
  \frametitle{Installation and Example}
\end{frame}
\subsection{Machine Learning}
\begin{frame}
  \frametitle{Python for Machine Learning}
  \begin{itemize}
    \item scikit-learn
    \item pyml
  \end{itemize}
\end{frame}

\begin{frame}
  \frametitle{scikit-learn}
  PyML is an interactive object oriented framework for machine learning written
   in Python. PyML focuses on SVMs and other kernel methods.
\end{frame}

\subsection{Network Visualization and Analysis}
\begin{frame}
  \frametitle{igraph for network visualization}
\end{frame}



\section{Bioinformatics in the Cloud using Python}

\subsection{Clouding Computing and Bioinformatics}

\begin{frame}{Clouding Computing}
  \begin{itemize}
    \item Amazon Web Service: aws.amazon.com
    \item Aliyun: aliyun.com
    \item Google Compute Engine: cloud.google.com
    \item Microsoft Azure: azure.microsoft.com
    \item \ldots
  \end{itemize}
\end{frame}

\begin{frame}{Bioinformatics in the Cloud}
  \begin{itemize}
    \item DNANexus: DNANexus.com
    \item SBGenomics: SBGenomics.com\\
      rabix: rabix.org
    \item GeneDock: GeneDock.com
    \item L3-Bioinformatics: l3-bioinfo.com
    \item tute genomics, Variant Analysis from Qiagen, \ldots
  \end{itemize}
\end{frame}

\subsection{Python SDK of DNANexus}

\begin{frame}
  \frametitle{Overview}
\end{frame}

\begin{frame}
  \frametitle{Installation}
\end{frame}

\subsection{Rabix from SBGenomics}

\begin{frame}
  \frametitle{Reproducible Research}
  \begin{block}{Reproducible Research}
The goal of reproducible research is to tie specific instructions to data
analysis and experimental data so that scholarship can be recreated, better understood and verified.
  \end{block}
  \begin{block}{References}
    \begin{itemize}
      \item https://www.coursera.org/course/repdata
      \item http://cran.r-project.org/web/views/ReproducibleResearch.html
    \end{itemize}
  \end{block}
\end{frame}

\begin{frame}
  \frametitle{Rabix Project}
  see rabix_ismb.pdf
\end{frame}

\end{document}
